\addcontentsline{toc}{chapter}{Abstract}

\chapter*{Abstract}
\hl{fix the abstract later}
Collaborative intelligence is an artificial intelligence deployment strategy that leverages edge-based and cloud-based compute resources to accelerate inference with deep models on edge devices. It entails splitting a deep model into a mobile device sub-model and a remote or cloud sub-model. Deep feature tensors originating from the device sub-model are transmitted over imperfect communication channels to the remote sub-model residing on the cloud. Burst losses in the deep feature tensors may occur due to the imperfect communication channel between the edge and the cloud. Corrupted tensors compromise the cloud sub-model's performance in completing the inference task.

In this work, we concern ourselves with the image classification task with a \gls{vgg16} and a \gls{resnet18}. Tensor completion methods may be employed to attempt to recover the missing packets in deep feature tensors. This report investigates tensor completion with two general methods from the literature, namely \gls{silrtc} and \gls{halrtc}, and a new, proposed method, named \gls{caltec}. Monte Carlo runs are executed to simulate Gilbert-Elliott channel realizations covering a range of burst loss probabilities and average burst lengths. These three tensor completion methods are evaluated on their performance in recovering the missing deep feature tensor values. When the loss probability is high (20\% and 30\%), the proposed method leads to the highest cloud prediction accuracy on \gls{resnet18} corrupted tensors. At lower loss probabilities of 1\% and 10\%, \gls{halrtc} edges ahead of \gls{caltec} by less than 1\% better cloud prediction accuracy. In all scenarios, \gls{caltec} is at least 8 times faster than the two general tensor completion methods. This makes \gls{caltec} ideal for applications which are sensitive to inference latency. This work also reports on updating a deep feature transmission simulator, \gls{dfts}, for compatibility with TensorFlow 2. Experiments are also run to study the effect of quantization and channel losses on cloud prediction accuracy.

\textbf{Keywords:} Tensor completion, collaborative intelligence, deep feature transmission, deep learning, missing data imputation, tensor reconstruction, cloud computing.


\clearpage